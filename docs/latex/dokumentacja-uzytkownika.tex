\documentclass[polish,12pt]{aghthesis}

\usepackage[utf8]{inputenc}

\usepackage{graphicx}
\usepackage{hyperref}
\usepackage{xcolor}
\usepackage{adjustbox}
\usepackage{enumitem}

\graphicspath{ {img/} }

\definecolor{dark-red}{rgb}{0.5,0,0}
\definecolor{dark-green}{rgb}{0,0.5,0}
\definecolor{dark-blue}{rgb}{0,0,0.5}
\hypersetup{
    colorlinks,
    linkcolor={black},
    urlcolor={dark-blue},
    citecolor={dark-green}
}

\newcommand{\img}[3]{
  \begin{figure}
    \centering
    \includegraphics[width=0.9\textwidth]{#1}
    \caption{#2}
    \label{fig:#3}
  \end{figure}
}

\title{Rozszerzony pulpit serwera X działający w rozproszonych systemach komputerowych}
\author{Miłosz Łakomy, Marcin Radomski}
\supervisor{dr hab. inż. Krzysztof Boryczko, prof. nadzw. AGH}



\begin{document}
\subtitle{Dokumentacja użytkownika}
\maketitle

\tableofcontents
\vfill
\pagebreak

\section{Skrypt klienta}

  Uruchamia klienta eXtend, który po połączeniu z serwerem wyświetla na ekranie część pulpitu serwera. \\

  \textbf{Sposób użycia:}
  \texttt{./client.py [ opcje... ]}

  \subsection{Dostępne opcje}
  \begin{itemize}
    \item \texttt{-h, -{}-help}

      Wyświetla listę dostępnych opcji wraz z ich opisem, po czym kończy działanie.

    \item \texttt{-g \emph{ADRES\_GRUPY}, -{}-multicast-group \emph{ADRES\_GRUPY}}

      Ustawia adres grupy multicastowej, na którą serwer wysyła koordynaty myszy. W przypadku pominięcia tej opcji użyta jest wartość zmiennej środowiskowej \emph{EXTEND\_MCAST\_GROUP} lub 224.0.126.93 jeżeli zmienna \emph{EXTEND\_MCAST\_GROUP} nie jest ustawiona.

    \item \texttt{-p \emph{PORT}, -{}-port \emph{PORT}}

      Ustawia numer portu używanego do komunikacji z serwerem. W przypadku pominięcia tej opcji użyta zostanie wartość zmiennej środowiskowej \emph{EXTEND\_PORT} lub 0x7e5d (32349) jeżeli zmienna \emph{EXTEND\_PORT} nie jest ustawiona.

    \item \texttt{-l \emph{PLIK\_LOCK}, -{}-lock-file \emph{PLIK\_LOCK}}

      Ustawia ścieżkę do pliku \emph{lock}. W przypadku pominięcia tej opcji plik ten będzie miał nazwę \emph{.eXtend-client.lock} i zostanie umieszczony w katalogu domowym użytkownika (określonym za pomocą zmiennej środowiskowej \emph{HOME}) lub \emph{/tmp/} jeżeli zmienna \emph{HOME} nie jest ustawiona.

    \item \texttt{-s \emph{ADRES\_SERWERA}, -{}-server-ip \emph{ADRES\_SERWERA}}

      Podczas startu próbuje połączyć się z serwerem o podanym adresie, zamiast łączyć się z nadawcą pierwszej otrzymanej wiadomości multicastowej.

    \item \texttt{-v \emph{KOMENDA\_KLIENTA\_VNC}, -{}-vnc-client-cmd \emph{KOMENDA\_KLIENTA\_VNC}}

       Ustawia komendę shella używaną do uruchomienia klienta VNC. Słowa \emph{HOST} i \emph{PORT} zostaną zastąpione odpowiednio adresem IP i portem używanym do połączenia do serwera VNC. Wartość domyślna: \texttt{vncviewer -fullscreen -viewonly HOST::PORT -autopass}.

    \item \texttt{-w \emph{PLIK}, -{}-vnc-passwd-file \emph{PLIK}}

      Ustawia ścieżkę do \emph{PLIK}u zawierającego hasło używane do autentykacji w serwerze VNC. W przypadku pominięcia tej opcji zostanie użyty nowo utworzony w katalogu domowym użytkownika plik \emph{.eXtend\_vncPwd} z domyślną zawartością \emph{lubieplacki}.

    \item \texttt{-i \emph{INTERFEJS}, -{}-interface \emph{INTERFEJS}}

      Ustawia nazwę interfejsu sieciowego, na którym klient nasłuchuje komunikatów multicastowych. W przypadku pominięcia tej opcji klient będzie nasłuchiwał na wszystkich dostępnych interfejsach.

  \end{itemize}

  \subsection{Opis działania}
    Skrypt klienta po uruchomieniu zaczyna nasłuchiwać na komunikaty multicastowe wysyłane na skonfigurowany \emph{ADRES\_GRUPY} i \emph{PORT}. Po otrzymaniu pierwszej wiadomości interpretuje nadawcę jako serwer eXtend i próbuje połączyć się z \emph{PORT}em nadawcy po TCP w celu przeprowadzenia autokonfiguracji. W przypadku pomyślnego nawiązania połączenia otrzymuje identyfikator, który jest wyświetlany w terminalu w celu ułatwienia ręcznej konfiguracji (patrz: \ref{sec:ManualConfig} Ręczna konfiguracja układu ekranów). Po zakończeniu konfiguracji po stronie serwera klient uruchamia w trybie pełnoekranowym aplikację klienta VNC, która rozpoczyna wyświetlanie fragmentu pulpitu serwera. Od tego momentu klient tylko steruje pozycją kursora, przesuwając go na odpowiednie współrzędne otrzymane z serwera (uwzględniając własną pozycję względem lewego górnego rogu pulpitu serwera).

    W przypadku utraty połączenia klient wraca w stan nasłuchiwania na komunikaty multicastowe. Może on też zrestartować klienta VNC, jeśli układ ekranów klientów ulegnie zmianie.

    Aby zapobiec przypadkowemu uruchomieniu kilku procesów klienta jednocześnie, podczas uruchamiania sprawdzana jest obecność pliku \emph{PLIK\_LOCK}, który jest usuwany przy wyjściu z programu.


\section{Skrypt serwera}

  Umożliwia uruchomienie i konfigurację serwera oraz zarządzanie podłączonymi klientami. \\

  \textbf{Sposób użycia:}
  \texttt{./server.py [ opcje... ]}

  \subsection{Dostępne opcje}

  \begin{itemize}
    \item \texttt{-h, -{}-help}

      Wyświetla listę dostępnych opcji wraz z ich opisem, po czym kończy działanie.

    \item \texttt{-s, -{}-start}

      Uruchamia w tle proces serwera, który rozpoczyna transmisję koordynatów myszy na ustawiony adres multicastowy i nasłuchuje przychodzących połączeń od klientów.

    \item \texttt{-S, -{}-stop}

      Zatrzymuje proces serwera.

    \item \texttt{-i \emph{INTERFEJS}, -{}-interfaces \emph{INTERFEJS}}

      Ustawia interfejs sieciowy, na który będą wysyłane komunikaty multicastowe. Można użyć tej opcji wielokrotnie, aby nadawać na kilku interfejsach. W przypadku pominięcia tej opcji komunikaty będą wysyłane na domyślny interfejs sieciowy.

    \item \texttt{-p \emph{PORT}, -{}-port \emph{PORT}}

      Ustawia numer portu używanego do przyjmowania połączeń klientów, jak również docelowego portu komunikatów multicastowych. W przypadku pominięcia tej opcji zostanie użyta magiczna liczba 0x7e5d (32349).

    \item \texttt{-P \emph{PLIK\_HASŁA}, -{}-password-file \emph{PLIK\_HASŁA}}

      Ustawia ścieżkę do pliku z hasłem przekazywanego do serwera VNC. Jego zawartość zostanie użyta jako hasło dostępu do VNC; w przypadku pominięcia tej opcji zostanie użyty nowo utworzony w katalogu domowym użytkownika plik \emph{.eXtend\_vncPwd} z domyślną zawartością \emph{lubieplacki}.

    \item \texttt{-l \emph{PLIK\_Z\_LOGAMI}, -{}-log-file \emph{PLIK\_Z\_LOGAMI}}

      Ustawia plik zawierający logi generowane przez proces serwera. W przypadku pominięcia tej opcji zostanie użyty plik \emph{.eXtend-server\_0X\textbf{NUMER\_PORTU}.log} w katalogu domowym użytkownika.

    \item \texttt{-m, -{}-manual-arrange}

      Wymuś ręczną konfigurację pozycji ekranów klientów (przy użyciu \texttt{./server.py -a}). W przypadku pominięcia tej opcji, kolejni podłączający się klienci będą układani na prawo od istniejących ekranów

    \item \texttt{-a \emph{UKŁAD}, -{}-arrange \emph{UKŁAD}}

      Ustawia ekran klienta o podanym identyfikatorze na odpowiednich współrzędnych. Przykładowo: \texttt{./server.py -a '0 123 456'} spowoduje, że klient o identyfikatorze \emph{0} zacznie wyświetlać fragment ekranu, którego lewy górny róg znajduje się na współrzędnych \emph{123, 456}. Rozmiar wyświetlanego fragmentu jest zależny od rozdzielczości monitora klienta.

    \item \texttt{-lu, -{}-list-unarranged}

      Wypisuje listę podłączonych klientów, którym jeszcze nie zostały przyporządkowane żadne współrzędne ekranu.

    \item \texttt{-la, -{}-list-arranged}

      Wypisuje listę podłączonych i skonfigurowanych klientów, którzy wyświetlają obraz przyporządkowanych im fragmentów pulpitu.

  \end{itemize}

  \subsection{Opis działania}
    Uruchomienie serwera z opcją \texttt{-{}-start} spowoduje utworzenie działającego w tle procesu obsługującego połączenia klientów. Jednocześnie serwer zaczyna rozgłaszać swoją obecność w sieci lokalnej, wysyłająć na skonfigurowane \emph{INTERFEJS}y komunikaty multicastowe.

    \textbf{UWAGA}: wysyłane komunikaty multicastowe mają ustawioną na sztywno wartość TTL równą 2.

    Po przyjęciu połączenia klienta automatycznie jest mu przypisywany unikalny identyfikator oraz współrzędne wyświetlanej przez niego części pulpitu. Z każdym przychodzącym połączeniem tworzony jest też kolejny wirtualny ekran, symulujący podłączenie monitora o rozdzielczości równej tej deklarowanej przez klienta. Automatyczna konfiguracja pozycji ekranów zakłada, że każdy następny jest ustawiany na prawo od istniejących ekranów.
    W przypadku utraty połączenia z klientem przyporządkowany mu wirtualny ekran jest usuwany.

    \subsubsection{\label{sec:ManualConfig}Ręczna konfiguracja układu ekranów}
      Uruchomienie serwera z opcją \texttt{-{}-manual-arrange} powoduje zawieszenie wątku obsługującego połączenie klienta w oczekiwaniu na przypisanie mu współrzędnych. Ustawia się je za pomocą opcji \texttt{-{}-arrange}. \texttt{-{}-list-unarranged} pozwala wyświetlić listę klientów oczekujących na konfigurację.

\end{document}
