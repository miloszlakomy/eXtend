\documentclass[polish,12pt]{aghthesis}

\usepackage[utf8]{inputenc}

\usepackage{graphicx}
\usepackage{hyperref}
\usepackage{xcolor}
\usepackage{adjustbox}
\usepackage{enumitem}

\graphicspath{ {img/} }

\definecolor{dark-red}{rgb}{0.5,0,0}
\definecolor{dark-green}{rgb}{0,0.5,0}
\definecolor{dark-blue}{rgb}{0,0,0.5}
\hypersetup{
    colorlinks,
    linkcolor={black},
    urlcolor={dark-blue},
    citecolor={dark-green}
}

\newcommand{\img}[3]{
  \begin{figure}
    \centering
    \includegraphics[width=0.9\textwidth]{#1}
    \caption{#2}
    \label{fig:#3}
  \end{figure}
}

\title{Rozszerzony pulpit serwera X działający w rozproszonych systemach komputerowych}
\author{Miłosz Łakomy, Marcin Radomski}
\supervisor{dr hab. inż. Krzysztof Boryczko, prof. nadzw. AGH}



\begin{document}
\maketitle

\tableofcontents
\vfill
\pagebreak

\section{Problem}

Problem, z którym zmagaliśmy się w ramach pracy inżynierskiej, to możliwość wykorzystania połączonych w sieć komputerów jako dodatkowych monitorów. Dodatkowym wymaganiem była możliwość uruchomienia projektu pod systemem Linux. Docelowo nasze rozwiązanie miało umożliwiać wyświetlenie pulpitu pojedynczego komputera na ekranie złożonym z 24 monitorów podłączonych do 4 niezależnych stacji roboczych dostępnym w Katedrze Informatyki AGH.

\section{Studium wykonalności i analiza ryzyka}

Już na etapie rozpoczęcia prac nad projektem wiedzieliśmy, że istnieją alternatywne rozwiązania, które jednak nie działają pod wymaganym systemem operacyjnym (Linux). Wiedzieliśmy zatem, że problem jest możliwy do rozwiązania, chociaż żaden z nas nie miał pojęcia, w jaki sposób powinniśmy to osiągnąć. Najwięcej wątpliwości co do wykonalności projektu brało się właśnie stąd, że pomimo istnienia analogicznych aplikacji pod inne systemy operacyjne nikt nie pokusił się o stworzenie wersji linuksowej. Pozwalało to domniemywać, że zagadnienie to może sprawiać więcej problemów, niż wydaje się na początku. Z tego powodu pierwsza iteracja została przez nas w całości poświęcona na zaznajomienie się ze sposobem działania podobnych rozwiązań oraz stworzenie minimalnego funkcjonalnego prototypu.

Kolejnym istotnym czynnikiem, jaki musieliśmy wziąć pod uwagę, jest różnorodność sprzętu, na którym można uruchomić Linuksa. Sterowniki kart graficznych różnych producentów nie udostępniają takich samych możliwości. Przekonaliśmy się o tym już podczas tworzenia prototypu, który uświadomił nam, że będziemy w stanie osiągnąć nasz cel wyłącznie na kartach graficznych Intela.

Nie mogliśmy też zapomnieć o wydajności - gdyby okazało się, że nasze rozwiązanie działa zbyt wolno lub jest bardzo wymagające obliczeniowo, mogłoby nie nadawać się do użytku.

\section{Proces}

Zastosowaliśmy iteracyjny model prac: w danym momencie nie skupialiśmy się na kompletnym rozwiązaniu całego projektu, ale wyznaczaliśmy sobie cel trochę przybliżający nas do ostatecznego produktu. Nie ustalaliśmy ścisłych terminów zakończenia iteracji, chociaż założyliśmy, że pojedyncza iteracja kończąca się funkcjonalnym prototypem nie powinna trwać dłużej niż 3-4 miesiące. Ponieważ repozytorium projektu znajduje się na Githubie, dla wygody użyliśmy mechanizmu "issues" do zarządzania zadaniami do wykonania.

Aby zapewnić zgodność powstającego produktu z oczekiwaniami klienta, kontaktowaliśmy się z nim na bieżąco, prezentując mu nasze pomysły i aktualny stan prac nad projektem. Pozwoliło to na szybkie wyjaśnianie pojawiających się wątpliwości i jasne sprecyzowanie wymagań.

Pierwszą iterację poświęciliśmy w całości na dogłębne zapoznanie się z problemem, analizę sposobu działania alternatywnych aplikacji oraz stworzenie działającego prototypu. Dzięki temu upewniliśmy się, że opracowane przez nas rozwiązanie jest wystarczająco dobre.

Drugą iterację spędziliśmy na doprowadzaniu prototypu do bardziej użytecznej postaci. Dodatkowo byliśmy zmuszeni usunąć problem w aplikacji x11vnc, napotkany przez nas podczas pierwszej iteracji.

Trzecia iteracja zaowocowała w pełni funkcjonalnym narzędziem terminalowym, będącym w stanie automatycznie nawiązywać połączenie klienta z serwerem oraz ponawiać je, gdyby zostało ono przerwane. Przyglądaliśmy się też możliwości całkowitego wyeliminowania zależności klientów.

Czwartą, ostatnią iterację poświęcliśmy na doszlifowywanie naszej aplikacji oraz uzupełnianie dokumentacji projektu.

\section{Iteracja 1}
  \subsection{Analiza dostępnych rozwiązań}
    Prace rozpoczęliśmy od przeszukania zasobów Internetu w celu znalezienia już istniejących aplikacji o funkcjonalności podobnej do tej, jaką chcieliśmy osiągnąć. Okazało się, że nie jest dostępny program w pełni spełniający oczekiwania naszego klienta. Trafiliśmy jednak na kilka projektów działających w sposób mniej lub bardziej zbliżony do wymaganego:

    \begin{itemize}
      \item \textbf{MaxiVista} - \url{http://maxivista.com}
        Jest to projekt bliźniaczo podobny do naszego. Umożliwia podpięcie dodatkowych komputerów po sieci z użyciem TCP/IP. Nie satysfakcjonuje on nas ze względu na pewne poważne ograniczenia:
        \begin{itemize}
          \item na maszynie będącej serwerem musi być zainstalowany system Windows, klientem może być Windows lub Mac
          \item można podłączyć maksymalnie 3 klienty
          \item większość opcji jest dostępna tylko w wersji Pro
          \item za darmo dostępna jest tylko 30-dniowa wersja próbna
          \item ilość dostępnych rozdzielczości jest mocno ograniczona
        \end{itemize}

      \item \textbf{ZoneScreen} - \url{http://zoneos.com/zonescreen.htm}
        Projekt analogiczny do MaxiVista, ale darmowy. Serwer działa tylko pod Windowsem, ale klientem może być dowolny system, na którym można uruchomić klienta VNC. Obsługuje więcej niż jednego klienta - autor nie podaje górnego ograniczenia.

      \item \textbf{AirDisplay} - \url{http://avatron.com/applications/air-display}
        Podobny do MaxiVista i ZoneScreen. Można podłączyć maksymalnie 4 ekrany, chociaż informacja na stronie mówi, że działa to tylko na Macach. Klientami mogą być urządzenia z Androidem, Windowsem, OS X-em lub iOSem.

      \item \textbf{Xdmx} - Distributed Multihead X - \url{http://dmx.sourceforge.net}, \url{http://cgit.freedesktop.org/xorg/lib/libdmx/}
        Open-sourcowy projekt oparty na hierarchii serwerów X - jeden "główny" (front-end) i wiele "podrzędnych" (back-end). Był to projekt, z którym początkowo wiązaliśmy największe nadzieje: w końcu był stworzony z myślą o linuksie i X11. Próby jego użycia jednak spełzły na niczym - okazało się, że projekt na Sourceforge nie jest aktualizowany od 10 lat i brakuje informacji o tym, że jego repozytorium zostało przeniesione na freedesktop.org/xorg. Brakuje aktualnej dokumentacji, aktualizacje pojawiają się niezwykle rzadko, a co gorsza sam projekt do obsługi wielu monitorów korzysta z rozszerzenia Xinerama, które zostało de facto zastąpione przez RandR. Mimo usilnych prób nie udało nam się doprowadzić Xdmx do działania.

      \item \textbf{ParaView} - \url{http://www.paraview.org/}
        ParaView jest obsługującą rozproszone wielomonitorowe konfiguracje aplikacją do wizualizacji danych w 3D. Jest to program używany w laboratorium w Katedrze Informatyki, ale niestety nie można użyć go jako narzędzia pozwalającego na transmisję fragmentu pulpitu.
    \end{itemize}

  \subsection{Analiza wykonalności i stworzenie prototypu}
    Analiza wspomnianych rozwiązań podsunęła nam pewien pomysł: gdybyśmy byli w stanie przekonać serwer X, że ma dostępną większą powierzchnię pulpitu niż to jest w rzeczywistości, moglibyśmy wysłać niewidoczną część pulpitu po sieci używając przykładowo VNC. Po przeszukaniu zasobów Internetu okazało się, że jest to jak najbardziej możliwe: standardowy sterownik grafiki dla kart Intela daje możliwość tworzenia wirtualnych monitorów (virtual output). Co prawda ogranicza nam to ilość sprzętu, na którym możemy uruchomić nasz projekt, jednak karty Intela są całkiem popularne i nie mieliśmy żadnych problemów ze znalezieniem odpowiedniego komputera.

    Po przebrnięciu przez sporą ilość dokumentacji, opcji konfiguracyjnych serwera X i samego sterownika xserver-xorg-video-intel udało nam się skonfigurować wirtualne pulpity na dwóch maszynach działających pod kontrolą systemu operacyjnego Ubuntu (w wersjach 13.04 i 13.10). Po tym zapoznaliśmy się z konfiguracją serwera VNC (x11vnc), co wystarczyło nam do stworzenia pierwszego prototypu aplikacji.

  \subsection{Spotkanie z klientem - podsumowanie iteracji 1}
    W połowie kwietnia spotkaliśmy się z klientem i zaprezentowaliśmy mu efekt naszych prac.

    Chociaż pojawiło się kilka nieoczekiwanych błędów, prezentację prototypu uznaliśmy za umiarkowany sukces. Działał on dużo lepiej niż się tego spodziewaliśmy, więc w tym momencie zdecydowaliśmy się kontynuować prace nad tym rozwiązaniem. Największym problemem w tym momencie były regularnie pojawiające się czarne prostokąty, widoczne na zdjęciach powyżej. Poza tym, szybkość odświeżania kursora myszy była niewystarczająca - dowiedzieliśmy się, że żeby aplikacja nie utrudniała pracy, potrzebujemy innego niż VNC mechanizmu aktualizacji jego pozycji.

    \vfill
    \img{it1_1.jpg}{[iteracja 1] test transmisji wideo}{It1VideoTest}
    \img{it1_2.jpg}{[iteracja 1] główna część pulpitu widoczna na ekranie laptopa}{It1Test1}
    \img{it1_3.jpg}{[iteracja 1] kontynuacja pulpitu laptopa na górnej części 24-segmentowego ekranu}{It1Test2}
    \img{it1_4.jpg}{[iteracja 1] 18 z 24 monitorów podłączonych do laptopa. W tym przypadku obraz z lewej górnej części ekranu z jakiegoś powodu został zduplikowany jako tapeta lewej dolnej części, co jednak nie przeszkadzało w użyciu.}{It1Test3}
    \pagebreak

\section{Iteracja 2}

  Podczas drugiej iteracji skupiliśmy się na wyeliminowaniu problemu czarnych prostokątów i uproszczeniu konfiguracji. Dotąd wszystko wymagało ręcznego uruchamiania aplikacji z terminala. Dodatkowo, przyjrzeliśmy się możliwościom Guacamole - aplikacji pozwalającej użyć przeglądarki internetowej jako klienta VNC.

  \subsection{Problem czarnych obszarów w x11vnc}

    W sytuacji, gdy cały pulpit złożony z ekranu laptopa (Output 0) oraz wirtualnych outputów wygląda tak jak na rysunku \ref{fig:It2VncBug1}, zaznaczony na czerwono obszar obserwowany na kliencie wyświetla się jako czarny prostokąt. Efekt ten jest przedstawiony na rysunku \ref{fig:It2VncBugScreen1}.

    \vfill
    \img{it2_vnc_bug_screen1.png}{W lewym górnym rogu maszyna wirtalna, odbierająca przez VNC obraz z kwadratu o rozmiarze 2000x2000 widocznego po prawej stronie.}{It2VncBugScreen1}
    \img{it2_vnc_bug1.png}{Konfiguracja sprawiająca problemy - obraz na obszarze zaznaczonym na czerwono nie wyświetla się.}{It2VncBug1}
    \img{it2_vnc_bug2.png}{Kolejna problematyczna konfiguracja}{It2VncBug2}
    \img{it2_vnc_bug3.png}{Kolejna problematyczna konfiguracja}{It2VncBug3}
    \img{it2_vnc_bug_screen2.png}{Problemy wydają się występować również w przypadku wysyłania obrazu strumieniem MPEG2-TS.}{It2VncBugScreen2}
    \img{it2_vnc_bug_screen3.png}{Układ outputów w niewykorzystanej przestrzeni wpływa na kształt niewidocznego obszaru.}{It2VncBugScreen3}
    \img{it2_vnc_bug_screen4.png}{Układ outputów w niewykorzystanej przestrzeni wpływa na kształt niewidocznego obszaru.}{It2VncBugScreen4}

    Obserwacje:
    \begin{itemize}
      \item prostokąt wydaje się mieć szerokość taką, jak Output 0, a jego górna krawędź jest na tej samej wysokości co dolna krawędź Outputu 0,
      \item problemy z tym obszarem pojawiają się nie tylko przy wysyłaniu obrazu przez VNC, ale też w przypadku MPEG2-TS,
      \item instalacja najnowszych sterowników Intela z repozytorium Xorg nie pomogła,
      \item utworzenie wirtualnego outputu na nieużywanym obszarze powoduje zniknięcie części czarnego prostokąta. Potencjalny workaround - dodać wirtualny output pokrywający cały nieużywany obszar.
    \end{itemize}
    \vfill

    Wygląda na to, że niezależnie od współrzędnych przesyłanego wycinka, prostokąt o szerokości równej szerokości Outputu 0 i offsecie 0x<output0\_y\_resolution> względem lewego górnego roku wycinka ma problemy z wyświeltaniem. Co ciekawe, obszar który teoretycznie nie jest przydzielony do żadnego outputu, a nie znajduje się w feralnym prostokącie, \textbf{jest} streamowany - widać taskbar Gnome, którego być tam nie powinno, a dodatkowo okna nie mają problemu z wyświetlaniem:

    Zastanawiające jest to, że screenshoty działają bez zarzutu. Przeanalizowaliśmy część kodu x11vnc i stworzyliśmy prostą aplikację korzystającą z funkcji XGetImage/XGetSubImage, wykorzystywanych również przez serwer VNC. Od strony sterownika wszystko wydaje się być bez zarzutu - funkcje zwracają to co powinny. Dalsza analiza kodu x11vnc pokazała, że serwer VNC z jakiegoś powodu nie chce oznaczać fragmentów ekranu z feralnego czarnego prostokąta jako obszar, który powinien być odrysowany. W najlepszym wypadku wysyła on jedynie małe kawałki, które pokazują się przy przesuwaniu okien nad problematycznym obszarem.

    Po jakichś dwóch tygodniach poszukiwania zakończyły się sukcesem. Okazało się, że x11vnc zaznacza obszary znajdujące się poza poprawnym obszarem pulpitu i pomija ich wysyłanie. Niestety, przy wyznaczaniu ich współrzędnych kompletnie ignoruje wartości podane przez używany przez nas parametr \emph{-clip}! W połączeniu z tym, że w całym x11vnc używane są współrzędne względem clipping regionu spowodowało widoczne powyżej problemy.

    Stworzyliśmy poprawkę, dostępną pod adresem \url{https://gist.github.com/dextero/bd2e22292a08417673e4} i na repozytorium projektu (plik \emph{x11vnc-blackout-clip-fix.patch}).

    Wysłaliśmy pull request z opisem problemu i opracowaną poprawką do autorów x11vnc. Po kilku dniach został on zaakceptowany, a łatka trafiła do głównej gałęzi repozytorium projektu. Można to zobaczyć pod adresem \url{https://github.com/LibVNC/libvncserver/pull/4}.

    W połowie maja przetestowaliśmy załataną wersję serwera VNC w laboratorium 1.18. Nie zaobserwowaliśmy występującego wcześniej błędu: czarne prostokąty bezpowrotnie zniknęły. Rzuciły się nam za to w oczy problemy z wydajnością przesyłania dużej powierzchni pulpitu z wysoką częstotliwością.

    \vfill
    \img{it2_1.jpg}{[iteracja 2] test załatanej wersji x11vnc}{It2Test1}
    \img{it2_2.jpg}{[iteracja 2] test załatanej wersji x11vnc}{It2Test2}
    \pagebreak

  \subsection{Uproszczenie konfiguracji}

    W celu uproszczenia konfiguracji powstał skrypt basha, automatyzujący konfigurację wirtualnych ekranów i uruchamianie serwera x11vnc. Eksperymentowaliśmy również ze stworzeniem prostego interfejsu graficznego do obsługi tego skryptu, który jednak później został porzucony z powodu niewielkiej przydatności w porównaniu do ilości wymaganych do zainstalowania bibliotek.

    \img{it2_3.jpg}{[iteracja 2] test graficznego interfejsu konfiguracyjnego}{It2Test3}
    \img{it2_4.jpg}{[iteracja 2] test graficznego interfejsu konfiguracyjnego}{It2Test4}

  \subsection{Guacamole - przeglądarkowy klient VNC}

    Użycie przeglądarki internetowej w charakterze klienta VNC otworzyłoby przed nami duże możliwości: bylibyśmy w stanie zamienić dowolne urządzenie z w miarę nowoczesną przeglądarką w klienta naszej aplikacji. Pod koniec iteracji 2 zdołaliśmy skonfigurować Guacamole (\url{http://guac-dev.org}) i przeprowadzić kilka prostych testów. Wyniki były obiecujące, więc w kolejnej iteracji poświęciliśmy trochę czasu na rozwinięcie projektu w tym kierunku.

    \vfill
    \img{it2_5.jpg}{[iteracja 2] Test Guacamole - ekran wyboru połączenia VNC}{It2Guac1}
    \img{it2_6.jpg}{[iteracja 2] Test Guacamole - obraz z VNC widoczny w przeglądarce}{It2Guac2}
    \pagebreak

\section{Iteracja 3}

  \subsection{Automatyczna konfiguracja}

    W tej iteracji zajęliśmy się stworzeniem skryptów umożliwiających proste podłączanie komputerów klienckich. Zaprojektowaliśmy system składający się z aplikacji klienckiej oraz serwerowej, umożliwiający automatyczne wykrywanie obecności serwera projektu w sieci oraz konfigurację połączenia z nim.

    \textbf{Aplikacja kliencka:}
    \begin{itemize}[noitemsep,nolistsep]
      \item wykrywa ruch multicastowy pochodzący od serwera w celu jego zlokalizowania,
      \item łączy się z wykrytym serwerem,
      \item używa klienta VNC do wyświetlenia części pulpitu serwera.
    \end{itemize}

    \textbf{Aplikacja serwerowa:}
    \begin{itemize}[noitemsep,nolistsep]
      \item rozgłasza swoją obecność w sieci komunikatami multicastowymi,
      \item przyjmuje połączenia klientów,
      \item przydziela każdemu klientowi część pulpitu, powiększając go w razie potrzeby,
      \item dla każdego klienta uruchamia serwer VNC wysyłający mu obraz z przydzielonej części pulpitu.
    \end{itemize}

  \subsection{Problem z odświeżaniem pozycji kursora myszy}

    Podczas spotkania z klientem okazało się, że częstotliwość odświeżania pozycji kursora myszy za pomocą VNC jest niewystarczającado komfortowej pracy. Po przeanalizowaniu dostępnych możliwości zdecydowaliśmy się zaimplementować własny mechanizm do transmisji pozycji kursora. Osiągnęliśmy to przez dołączenie tej informacji do pakietów multicastowych wysyłanych przez serwer.

  \subsection{Prototyp wersji przeglądarkowej}

    W trakcie tej iteracji powstał również prototyp modułu serwera umożliwiający podłączanie klientów przeglądarkowych z użyciem Guacamole. Został on zaprojektowany w taki sposób, aby mógł być używany równolegle z częścią obsługująca standardowe  klienty. Implementacja okazała się być jednak nietrywialna i powstały prototyp nie działał na tyle niezawodnie, żeby mógł stać się częścią projektu. Kolejnym z problemów było znaczne zwiększenie zależności aplikacji serwerowej. Po konsultacji z klientem i ustaleniu, że nie jest to niezbędna część aplikacji postanowiliśmy pozostawić ten moduł w fazie prototypu, nie dodając go do finalnej wersji aplikacji.

\section{Iteracja 4}

  Ostatnią iterację przeznaczyliśmy na doszlifowanie projektu: usunęliśmy kilka drobnych błędów i uzupełniliśmy dokumentację.

\section{Podział prac}

  \textbf{Prace wykonane wspólnie}
  \begin{itemize}[noitemsep]
    \item analiza istniejących rozwiązań rozwiązujących podobny problem
    \item opracowanie modelu komunikacji klient-serwer umożliwiającego automatyczną konfigurację serwera
    \item opracowanie rozwiązania usprawniającego synchronizację ruchu kursora na ekranie klienta
    \item dokumentacja projektu
  \end{itemize}

  \noindent
  \textbf{Miłosz Łakomy}
  \begin{itemize}[noitemsep]
    \item opracowanie prototypowej wersji aplikacji
    \item stworzenie skryptów upraszczających konfigurację wczesnej wersji projektu
    \item zaprojektowanie i implementacja skryptu serwera
  \end{itemize}

  \noindent
  \textbf{Marcin Radomski}
  \begin{itemize}[noitemsep]
    \item zlokalizowanie błędu w x11vnc i stworzenie łatki
    \item zaprojektowanie i implementacja skryptu klienta
    \item opracowanie prototypu klienta przeglądarkowego
  \end{itemize}

  \vfill

\end{document} 
